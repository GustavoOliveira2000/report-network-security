\PassOptionsToPackage{usenames,dvipsnames,table}{xcolor}
\documentclass[sigconf,balance,nonacm,authordraft]{acmart}

% packages
\usepackage[utf8]{inputenc}
\usepackage{amsmath}
\usepackage{commath}

\usepackage{graphicx}
\usepackage[caption=false,font=footnotesize]{subfig}
    
\usepackage{cleveref}

% algorithms
\usepackage{algorithm,algpseudocode}
 
% For syntax highlighting
\usepackage{specs}

\usepackage{soul}
\usepackage{booktabs}
\usepackage{multirow}
\usepackage[nolist,nohyperlinks]{acronym}
\usepackage{xspace}
\usepackage{url}

\usepackage{pgfplots}
\pgfplotsset{width=10cm,compat=1.9} 


% document
\begin{document}

% acronyms
\begin{acronym}
\acro{DSL}{Domain-Specific Language}
\acro{HPC}{High-Performance Computing}
\acro{SSI}{Self-Sovereign Identity}
\acro{EUDI}{European Digital Identity}
\end{acronym}

% title and authors
\title{The Title of Your Work}

\author{Your Name}
\affiliation{%
  \institution{Department,\\University}
  \city{City} 
  \country{Country} 
  \postcode{Postal Code}
}
\email{Your Email}


\begin{abstract}
This is the abstract. Here, you should provide a short summary of the entire work. You should include the following elements: your research problem and objectives, your methods, your key results, and your conclusion. Aim for 150 to 300 words.
\end{abstract}

\keywords{Your, Keywords, Go, Here}

\maketitle

% % % % % % % % % % % % % % % % % % % % %
% 			INTRODUCTION
% % % % % % % % % % % % % % % % % % % % %
\section{Introduction}
\label{sec:intro}
This section serves as the introduction to the work and it is likely the second thing readers will focus on, after the abstract. This section should contextualize your work, introducing the topic and the expected results. Usually, it is also in this section that authors include a brief summary of the main contributions of the work.

Additionally, you can have several subsections for topics such as the problem statement, the motivation for this work, and also any needed background information for unfamiliar readers. These can also be made into their own sections that follow the introduction. The problem statement is a short description of the problem that your work addresses, identifying the gaps between the current state and the desired future state. The motivation should sell the importance of this work and the reader should understand what is the impact of solving the identified problem.

% % % % % % % % % % % % % % % % % % % % %
% 	    Architectural Analysis
% % % % % % % % % % % % % % % % % % % % %
\section{Architectural Analysis}
\label{sec:ArchitecturalAnalysis}
The European Digital Identity Wallet Architecture and Reference Framework (ARF) defines the technical and organizational foundation for a secure, interoperable, and privacy-preserving digital identity ecosystem across the European Union.
It provides a modular architecture that ensures trust and interoperability among Member States while giving users full control over their digital identity credentials.

According to Sections 4.1–4.3.1 of the ARF, the system is designed around a Wallet Unit that interfaces with external entities through standardized protocols.
The architecture follows key design principles such as user-centricity, security- and privacy-by-design, and cross-border interoperability, ensuring that all components interact coherently within the regulatory scope of eIDAS 2.0 (Regulation 2024/1183).

\subsection{Identity Management}
Identity management in the ARF centers on the secure issuance, storage, and presentation of digital credentials, with a strong emphasis on user control and privacy. The Wallet Unit acts as the primary interface for users to manage their digital identities, supporting the storage of various types of credentials, including those issued by public authorities and private entities. The ARF mandates the use of strong authentication mechanisms, such as multi-factor authentication (MFA), to access the wallet and to authorize transactions. Furthermore, identity proofing and credential issuance are performed in compliance with eIDAS 2.0 requirements, ensuring high levels of assurance and interoperability across Member States~\cite{EU_ARF2024,EU_eIDAS2024}. The architecture also provides mechanisms for selective disclosure, allowing users to share only the minimum necessary information with relying parties, thereby enhancing privacy by design.

\subsection{Trust Services}
Trust services are fundamental to establishing and maintaining confidence in digital transactions within the European Digital Identity ecosystem. The ARF leverages qualified trust service providers (QTSPs) to deliver essential services such as electronic signatures, seals, timestamps, and electronic attestation of attributes. These services are integrated into the Wallet Unit, enabling users to sign documents, authenticate transactions, and verify the integrity and origin of data. The ARF ensures that trust services comply with the legal and technical requirements set out in eIDAS 2.0, including the use of qualified certificates and secure signature creation devices~\cite{EU_ARF2024,EU_eIDAS2024}. This integration supports cross-border recognition of trust services, facilitating seamless digital interactions throughout the EU.

\subsection{Data Exchange Protocols}
The ARF specifies standardized data exchange protocols to guarantee secure and interoperable communication between the Wallet Unit and external entities, such as issuers, verifiers, and other wallets. Protocols are based on open standards, including OpenID Connect (OIDC), SAML, and verifiable credential data models, which enable secure transmission of identity attributes and credentials. The framework mandates end-to-end encryption and mutual authentication to protect the confidentiality and integrity of exchanged data. Additionally, the protocols support consent management, ensuring that users are always informed and in control of the data shared with third parties~\cite{EU_ARF2024}. These measures promote interoperability and trust among diverse national and sectoral systems.

\subsection{Summary}
In summary, the ARF provides a robust architectural foundation for the European Digital Identity Wallet, addressing critical aspects such as identity management, trust services, and secure data exchange. By adhering to the principles of security, privacy, and interoperability, and by aligning with the regulatory framework of eIDAS 2.0, the ARF aims to foster a trustworthy and user-centric digital identity ecosystem across the EU~\cite{EU_ARF2024,EU_eIDAS2024}.

% % % % % % % % % % % % % % % % % % % % %
%        SELF-SOVEREIGN PRINCIPLES
% % % % % % % % % % % % % % % % % % % % %
\section{Self-Sovereign Principles}
\label{sec:ssi}

Self-Sovereign Identity (SSI) represents a paradigm shift in digital identity management, placing individuals at the centre of control over their personal data. Unlike traditional centralised systems where identity providers hold and assert credentials on behalf of users, SSI enables individuals to maintain their own cryptographically signed attestations and selectively reveal attributes when interacting with services. The European Digital Identity (EUDI) Wallet adapts these foundational SSI principles within a legally binding regulatory framework to meet the requirements of eIDAS 2.0 (Regulation 2024/1183). This section examines how the EUDI Wallet implements core SSI concepts: user sovereignty, verifiable credentials, selective disclosure, and privacy-enhancing techniques. It also highlights the architectural and governance differences that distinguish the EUDI Wallet from decentralised SSI implementations.

\subsection{User Control and Data Sovereignty}

The principle of user sovereignty, whereby individuals control the storage, management, and disclosure of their identity data, distinguishes SSI from traditional identity systems. In the EUDI ecosystem, credentials reside on the user's device rather than in centralised registries, and explicit consent is required before any attribute is shared with a relying party~\cite{ZKPDiscussion_Recital15,ZKPDiscussion_Article5a}. Citizens may request, select, store, delete, and share identity data at their discretion, and the wallet architecture prevents both issuers and verifiers from tracking user behaviour across transactions~\cite{ZKPDiscussion_Article5a}. 

Independent analyses confirm that the EUDI Wallet grants users complete control over their credentials, allowing them to decide what information to reveal, with whom, and when~\cite{Gataca_ARF_SSI,ISC2_PrivacyDataSovereignty}. This decentralised storage model ensures that personal data sovereignty remains with the holder rather than service providers. Under eIDAS 2.0, every Member State must provide at least one EUDI Wallet solution enabling citizens to access public and private services using their own verifiable credentials~\cite{PortoThesis_Requirements}. These provisions institutionalise data sovereignty as a fundamental right within the European digital identity ecosystem.

\subsection{Verifiable Credentials and Trust Frameworks}

The SSI model operates on a three-party trust relationship: issuers create digitally signed credentials, holders store them in their wallets, and verifiers accept cryptographic proofs of credential possession. The EUDI Architecture and Reference Framework (ARF) adopts this structure through Person Identification Data (PID) and Electronic Attestations of Attributes (EAAs), which are issued by qualified trust service providers under eIDAS 2.0~\cite{PortoThesis_Structure}. These attestations are cryptographically bound to the holder through device keys and can be independently verified without contacting the issuer.

Trust in this ecosystem is anchored through trust registries that allow verifiers to retrieve issuer metadata and validate certificate status, thereby supporting cross-border credential recognition~\cite{PortoThesis_Structure,Finextra_SSI_Components}. Crucially, the ARF does not mandate distributed ledger technologies. Instead, it relies on qualified trust service providers and public-private governance arrangements to establish trust~\cite{Finextra_SSI_Differences}. This approach distinguishes the EUDI Wallet from many blockchain-based SSI systems while maintaining the core verification benefits of cryptographically signed credentials.

\subsection{Selective Disclosure and Privacy-Enhancing Techniques}

Data minimisation, the practice of disclosing only the attributes necessary for a given transaction, is a cornerstone of SSI privacy protection. The ARF explicitly defines selective disclosure as the capability for the wallet to present a subset of user attributes from PID or EAA attestations~\cite{ARF_SelectDisclosureDefinition}. High-level requirements mandate that all PID and EAA credentials support selective disclosure through privacy-preserving formats such as selective-disclosure JSON Web Tokens (SD-JWT) or mobile security objects~\cite{ARF_AttestationSelectiveDisclosure}.

In practice, when a verifier requests proof of a specific attribute, the wallet constructs a presentation containing only the requested data. For example, a merchant can verify that a customer is over 18 years old without learning the customer's name, date of birth, or address~\cite{ISC2_SelectDisclosureExample,DocuSign_SelectDisclosure}. This capability reduces both privacy risks and fraud exposure while maintaining regulatory compliance.

The discussion on zero-knowledge proofs within the ARF emphasises the importance of privacy-preserving technologies that enable validators to confirm statements without accessing underlying personal data~\cite{ZKPDiscussion_Recital59}. Recital 59 of the regulation specifically calls for techniques like zero-knowledge proofs to validate claims while preserving privacy, and Article 5a mandates that wallets prevent correlation of presentations across different services~\cite{ZKPDiscussion_Recital59,ZKPDiscussion_Article5a}. These technical measures, combined with architectural requirements for unlinkability, ensure that data minimisation and privacy-by-design are embedded throughout the system~\cite{Potential_PrivacyDesign}.

\subsection{Interoperability and Security}

The EUDI Wallet architecture is guided by four foundational principles: user-centricity, interoperability, privacy by design, and security by design~\cite{ISC2_GuidingPrinciples}. User-centricity ensures that holders retain the authority to decide which credentials to present and may revoke consent at any time~\cite{ISC2_ConsentRole,Potential_UserCentricity}. Transparency regarding data sharing (what is shared, with whom, and for what purpose) is maintained throughout each transaction.

Interoperability is achieved through adherence to open standards such as OpenID Connect for verifiable presentations and the W3C verifiable credentials data model, combined with legal mechanisms under eIDAS 2.0 that ensure cross-border recognition of credentials and trust services~\cite{Finextra_SSI_Components}. This standardisation enables seamless interaction between national and sectoral systems across the European Union.

Privacy by design is operationalised through selective disclosure, zero-knowledge proofs, and architectural safeguards against tracking and linkability~\cite{ZKPDiscussion_Article5a,Potential_PrivacyDesign}. Security by design integrates strong authentication mechanisms, including multi-factor authentication, secure hardware for cryptographic key storage, and rigorous protocols for credential protection~\cite{ISC2_GuidingPrinciples,Potential_SecurityDesign}. Together, these principles adapt SSI concepts to a high-assurance, legally compliant framework that balances decentralisation with regulatory oversight.

\subsection{Differences from Decentralised SSI Implementations}

Although the EUDI Wallet incorporates core SSI principles, it diverges from pure decentralised identity systems in several significant ways. The eIDAS 2.0 architecture does not require blockchain or distributed ledger technologies; trust is instead anchored via qualified trust service providers operating under regulatory supervision~\cite{Finextra_SSI_Differences}. Credentials and electronic signatures issued within the EUDI ecosystem carry legal binding force under European law, providing a level of assurance not typically present in permissionless blockchain-based systems.

Furthermore, the EUDI Wallet operates within a hybrid public-private governance framework that balances the decentralisation benefits of SSI with government-mandated assurance levels and compliance requirements, including strong customer authentication under the Payment Services Directive (PSD2)~\cite{Finextra_SSI_Differences}. Standardised protocols and attestation formats ensure both interoperability and legal recognition. These architectural choices reflect a pragmatic adaptation of SSI principles to the regulatory and operational requirements of a continent-wide digital identity infrastructure.

\subsection{Summary}

The EUDI Wallet represents a synthesis of SSI principles and regulatory compliance, embedding user control, verifiable attestations, selective disclosure, and privacy-enhancing cryptography within a unified legal framework. By combining privacy-preserving techniques with trust registries and qualified service providers, the ARF enables European citizens to prove identity attributes across borders while maintaining sovereignty over their personal information~\cite{EU_ARF2024,EU_eIDAS2024}. The result is a hybrid identity system that empowers individuals, protects privacy, and fosters cross-border interoperability within the EU's digital single market.
% % % % % % % % % % % % % % % % % % % % %
% 			YOUR SOLUTION
% % % % % % % % % % % % % % % % % % % % %
\section{Your Solution}
\label{sec:solution}
In this section you will describe your work, both in form and function. That is, you should describe its structure/architecture, but also how it works, both internally and externally. Internally, any data structures and algorithms that you have used are relevant, especially if they were developed especially for this work. Externally, consider how an end-user would use your work.

Additionally, in this section you can also present the technical difficulties faced during the development, if and how they were overcome, and any limitations of the final product.

Artifacts such as code or API listings do not belong in this section. If they are important to the report and need to be included, they should be put in appendices at the end of the document.

% % % % % % % % % % % % % % % % % % % % %
% 			EXPERIMENTAL EVALUATION
% % % % % % % % % % % % % % % % % % % % %
\section{Experimental Evaluation}
\label{sec:evaluation}
In this section you can describe the experimental evaluation that your performed to evaluate your project.

\subsection{Experimental Setup}
In this subsection you present the setup you used to perform your tests. The benchmarks uses, the runtime system, the hardware, the methodology, and so on. This is needed for reproducibility purposes and for readers to understand how your work is more or less relevant in their own context.

\subsection{Experimental Results}
In this subsection you present the achieved results and you conduct an analysis of the experiments to see if the results obtained are what was expected according to your initial assumptions. If they are not, you are expected to understand and explain why.

% % % % % % % % % % % % % % % % % % % % %
% 			RELATED WORK
% % % % % % % % % % % % % % % % % % % % %
\section{Related Work}
\label{sec:rel}
In this section you can present other work that is relevant to your own research and development. Usually a short paragraph will explain how the other approach works and what are the main differences to your work, e.g., what problems were left open that you manage to solve or how you took a different approach and what are its pros and cons.

% % % % % % % % % % % % % % % % % % % % %
% 			CONCLUSIONS
% % % % % % % % % % % % % % % % % % % % %
\section{Conclusions}
\label{sec:conclusions}
In this section you present the main conclusions of your work in a summarized form. You can also present relevant future work on to to tackle current limitations and extend functionality.

% % % % % % % % % % % % % % % % % % % % %
%      USEFUL EXAMPLES (TO DELETE)
% % % % % % % % % % % % % % % % % % % % %
\section{Useful Examples}
\label{sec:examples}
This section, which you should delete later on, has examples on how to use some of Latex's most common features and environments.



\Cref{alg:memoi} presents a basic algorithm using the \textit{algorithmic} environment for pseudo-code.

This is how you make an unumbered list:
\begin{itemize}
    \item This is the first item;
    \item This is the second item;
    \item This is the final item.
    \item asd
\end{itemize}

Whenever you cite someone else's work, you need to include a reference to the relevant source. This is how you make a citation~\cite{Michie1968}. You can also cite multiple sources at once like this~\cite{Strachey2000,Connors2000}. The references need to be present in the \textit{refs.bib} file.



 shows you how to include a figure. By default, Latex uses PNG or PDF formats for figures.

If you need to include equations, you can use Latex's excellent math environments. Here is a simple example of an equation:
\begin{equation}\label{eq:log}
    O(i, j) = c \times \log{(1+I(i, j))},
\end{equation}
where $c$ is a constant, $I$ is the input, $O$ is the output, and $i$ and $j$ are the image coordinates.

You can use the acronym package to help manage acronyms. The first time you use an acronym, its full form will be displayed, e.g., \ac{DSL} or \ac{HPC}. However, the following times, only the short version will be used, as in \ac{DSL} or \ac{HPC}. You can also use the plural form of acronyms, e.g., \acp{DSL}. The list of known acronyms is defined in the preamble of the document.



If you want to include code, you can also do so by using the listings package. Here is an example of how to do it. \Cref{fig:code} presents an example of how to include source code.

Finally, tables can also be a good option to present data. \Cref{tab:table1} gives you an example of how to generate tables. Notice that the caption appears above the table, contrary to what happens with figures.

\begin{table}[b]
\caption{This is the table caption.}
\label{tab:table1}
\footnotesize
\begin{tabular}{lll}
\toprule
         & machine A                   & machine B                           \\
\midrule
CPU      & Intel Core i7-9700 CPU      & 2x Intel Xeon E5-2630 v3            \\
CPU Frequency& 3.00GHz                     & 2.40GHz                             \\
RAM      & 16GB DDR4                   & 128GB                               \\
OS       & Ubuntu 20.04 LTS            & Ubuntu 16.04 LTS                    \\
Compiler & GCC 9.3                     & GCC 7.3                             \\
libm     & v2.31                       & v2.23                               \\
libomp   & v4.5                        & v4.5                                \\
\bottomrule
\end{tabular}
\end{table}

% % % % % % % % % % % % % % % % % % % % %
% 			ACKNOWLEDGMENTS
% % % % % % % % % % % % % % % % % % % % %
\section*{Acknowledgments}
The author would like to acknowledge \ldots

% % % % % % % % % % % % % % % % % % % % %
% 			BIBLIOGRAPHY
% % % % % % % % % % % % % % % % % % % % %
\bibliographystyle{ACM-Reference-Format}
\bibliography{refs}

\end{document}
